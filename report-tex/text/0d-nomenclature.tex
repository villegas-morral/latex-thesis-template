\chapter*{Nomenclature}\label{ch:nomenclature}
\addcontentsline{toc}{chapter}{Nomenclature}
\markboth{Nomenclature}{}

%% It can be done with the nomencl package too.
% -----------------------------------------------------
% \usepackage[intoc]{nomencl} % nomenclature in ToC
% \makenomenclature
% \renewcommand{\nomname}{Symbols and Definitions}
% \renewcommand{\nompreamble}{This section provides a reference for the notation used throughout the project. \vspace{2em}}
% \setlength{\nomlabelwidth}{4em}
% -----------------------------------------------------

This section serves as a reference for the notation used throughout the text.
\vspace{1em}

\renewcommand{\arraystretch}{1.2}

{\small                                 % smaller font size for the table
\begin{xltabular}{\textwidth}{ 
    >{\centering\arraybackslash}p{0.12\textwidth}   % first column: 12% of text width
    >{\centering\arraybackslash}p{0.2\textwidth}
    X                                               % third column: take up the remaining space
    >{\centering\arraybackslash}p{0.11\textwidth}   % fourth column: 11% of text width
}
    \hline
    \textbf{Symbol} & \textbf{Code} & \multicolumn{1}{c}{\textbf{Description}} & \textbf{Value} \\      % header
    \hline
        $T_0$ & \code{t0} & Typical time scale & \SI{2}{\hour} \\
        $R_0$ & \code{r0} & Typical length scale & \SI{5}{\micro\meter} \\
    \hline
\end{xltabular}
}






